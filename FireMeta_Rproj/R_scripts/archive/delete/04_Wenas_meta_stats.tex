% Options for packages loaded elsewhere
\PassOptionsToPackage{unicode}{hyperref}
\PassOptionsToPackage{hyphens}{url}
%
\documentclass[
]{article}
\usepackage{amsmath,amssymb}
\usepackage{lmodern}
\usepackage{iftex}
\ifPDFTeX
  \usepackage[T1]{fontenc}
  \usepackage[utf8]{inputenc}
  \usepackage{textcomp} % provide euro and other symbols
\else % if luatex or xetex
  \usepackage{unicode-math}
  \defaultfontfeatures{Scale=MatchLowercase}
  \defaultfontfeatures[\rmfamily]{Ligatures=TeX,Scale=1}
\fi
% Use upquote if available, for straight quotes in verbatim environments
\IfFileExists{upquote.sty}{\usepackage{upquote}}{}
\IfFileExists{microtype.sty}{% use microtype if available
  \usepackage[]{microtype}
  \UseMicrotypeSet[protrusion]{basicmath} % disable protrusion for tt fonts
}{}
\makeatletter
\@ifundefined{KOMAClassName}{% if non-KOMA class
  \IfFileExists{parskip.sty}{%
    \usepackage{parskip}
  }{% else
    \setlength{\parindent}{0pt}
    \setlength{\parskip}{6pt plus 2pt minus 1pt}}
}{% if KOMA class
  \KOMAoptions{parskip=half}}
\makeatother
\usepackage{xcolor}
\usepackage[margin=1in]{geometry}
\usepackage{color}
\usepackage{fancyvrb}
\newcommand{\VerbBar}{|}
\newcommand{\VERB}{\Verb[commandchars=\\\{\}]}
\DefineVerbatimEnvironment{Highlighting}{Verbatim}{commandchars=\\\{\}}
% Add ',fontsize=\small' for more characters per line
\usepackage{framed}
\definecolor{shadecolor}{RGB}{248,248,248}
\newenvironment{Shaded}{\begin{snugshade}}{\end{snugshade}}
\newcommand{\AlertTok}[1]{\textcolor[rgb]{0.94,0.16,0.16}{#1}}
\newcommand{\AnnotationTok}[1]{\textcolor[rgb]{0.56,0.35,0.01}{\textbf{\textit{#1}}}}
\newcommand{\AttributeTok}[1]{\textcolor[rgb]{0.77,0.63,0.00}{#1}}
\newcommand{\BaseNTok}[1]{\textcolor[rgb]{0.00,0.00,0.81}{#1}}
\newcommand{\BuiltInTok}[1]{#1}
\newcommand{\CharTok}[1]{\textcolor[rgb]{0.31,0.60,0.02}{#1}}
\newcommand{\CommentTok}[1]{\textcolor[rgb]{0.56,0.35,0.01}{\textit{#1}}}
\newcommand{\CommentVarTok}[1]{\textcolor[rgb]{0.56,0.35,0.01}{\textbf{\textit{#1}}}}
\newcommand{\ConstantTok}[1]{\textcolor[rgb]{0.00,0.00,0.00}{#1}}
\newcommand{\ControlFlowTok}[1]{\textcolor[rgb]{0.13,0.29,0.53}{\textbf{#1}}}
\newcommand{\DataTypeTok}[1]{\textcolor[rgb]{0.13,0.29,0.53}{#1}}
\newcommand{\DecValTok}[1]{\textcolor[rgb]{0.00,0.00,0.81}{#1}}
\newcommand{\DocumentationTok}[1]{\textcolor[rgb]{0.56,0.35,0.01}{\textbf{\textit{#1}}}}
\newcommand{\ErrorTok}[1]{\textcolor[rgb]{0.64,0.00,0.00}{\textbf{#1}}}
\newcommand{\ExtensionTok}[1]{#1}
\newcommand{\FloatTok}[1]{\textcolor[rgb]{0.00,0.00,0.81}{#1}}
\newcommand{\FunctionTok}[1]{\textcolor[rgb]{0.00,0.00,0.00}{#1}}
\newcommand{\ImportTok}[1]{#1}
\newcommand{\InformationTok}[1]{\textcolor[rgb]{0.56,0.35,0.01}{\textbf{\textit{#1}}}}
\newcommand{\KeywordTok}[1]{\textcolor[rgb]{0.13,0.29,0.53}{\textbf{#1}}}
\newcommand{\NormalTok}[1]{#1}
\newcommand{\OperatorTok}[1]{\textcolor[rgb]{0.81,0.36,0.00}{\textbf{#1}}}
\newcommand{\OtherTok}[1]{\textcolor[rgb]{0.56,0.35,0.01}{#1}}
\newcommand{\PreprocessorTok}[1]{\textcolor[rgb]{0.56,0.35,0.01}{\textit{#1}}}
\newcommand{\RegionMarkerTok}[1]{#1}
\newcommand{\SpecialCharTok}[1]{\textcolor[rgb]{0.00,0.00,0.00}{#1}}
\newcommand{\SpecialStringTok}[1]{\textcolor[rgb]{0.31,0.60,0.02}{#1}}
\newcommand{\StringTok}[1]{\textcolor[rgb]{0.31,0.60,0.02}{#1}}
\newcommand{\VariableTok}[1]{\textcolor[rgb]{0.00,0.00,0.00}{#1}}
\newcommand{\VerbatimStringTok}[1]{\textcolor[rgb]{0.31,0.60,0.02}{#1}}
\newcommand{\WarningTok}[1]{\textcolor[rgb]{0.56,0.35,0.01}{\textbf{\textit{#1}}}}
\usepackage{graphicx}
\makeatletter
\def\maxwidth{\ifdim\Gin@nat@width>\linewidth\linewidth\else\Gin@nat@width\fi}
\def\maxheight{\ifdim\Gin@nat@height>\textheight\textheight\else\Gin@nat@height\fi}
\makeatother
% Scale images if necessary, so that they will not overflow the page
% margins by default, and it is still possible to overwrite the defaults
% using explicit options in \includegraphics[width, height, ...]{}
\setkeys{Gin}{width=\maxwidth,height=\maxheight,keepaspectratio}
% Set default figure placement to htbp
\makeatletter
\def\fps@figure{htbp}
\makeatother
\setlength{\emergencystretch}{3em} % prevent overfull lines
\providecommand{\tightlist}{%
  \setlength{\itemsep}{0pt}\setlength{\parskip}{0pt}}
\setcounter{secnumdepth}{-\maxdimen} % remove section numbering
\ifLuaTeX
  \usepackage{selnolig}  % disable illegal ligatures
\fi
\IfFileExists{bookmark.sty}{\usepackage{bookmark}}{\usepackage{hyperref}}
\IfFileExists{xurl.sty}{\usepackage{xurl}}{} % add URL line breaks if available
\urlstyle{same} % disable monospaced font for URLs
\hypersetup{
  pdftitle={04\_Wenas\_meta\_stats},
  hidelinks,
  pdfcreator={LaTeX via pandoc}}

\title{04\_Wenas\_meta\_stats}
\author{}
\date{\vspace{-2.5em}2023-07-11}

\begin{document}
\maketitle

\hypertarget{load-packages-and-set-working-directory}{%
\subsection{Load packages and set working
directory}\label{load-packages-and-set-working-directory}}

\begin{Shaded}
\begin{Highlighting}[]
\CommentTok{\#for Jake/mac}

\FunctionTok{rm}\NormalTok{(}\AttributeTok{list=}\FunctionTok{ls}\NormalTok{(}\AttributeTok{all=}\NormalTok{T)) }\CommentTok{\#this clears your Environment}


\FunctionTok{library}\NormalTok{(tidyverse)}
\end{Highlighting}
\end{Shaded}

\begin{verbatim}
## -- Attaching core tidyverse packages ------------------------ tidyverse 2.0.0 --
## v dplyr     1.1.2     v readr     2.1.4
## v forcats   1.0.0     v stringr   1.5.0
## v ggplot2   3.4.2     v tibble    3.2.1
## v lubridate 1.9.2     v tidyr     1.3.0
## v purrr     1.0.1     
## -- Conflicts ------------------------------------------ tidyverse_conflicts() --
## x dplyr::filter() masks stats::filter()
## x dplyr::lag()    masks stats::lag()
## i Use the conflicted package (<http://conflicted.r-lib.org/>) to force all conflicts to become errors
\end{verbatim}

\begin{Shaded}
\begin{Highlighting}[]
\FunctionTok{library}\NormalTok{(rstatix)}
\end{Highlighting}
\end{Shaded}

\begin{verbatim}
## 
## Attaching package: 'rstatix'
## 
## The following object is masked from 'package:stats':
## 
##     filter
\end{verbatim}

\begin{Shaded}
\begin{Highlighting}[]
\FunctionTok{library}\NormalTok{(ggpubr)}
\FunctionTok{library}\NormalTok{(ggplot2)}
\FunctionTok{library}\NormalTok{(coin)}
\end{Highlighting}
\end{Shaded}

\begin{verbatim}
## Loading required package: survival
## 
## Attaching package: 'coin'
## 
## The following objects are masked from 'package:rstatix':
## 
##     chisq_test, friedman_test, kruskal_test, sign_test, wilcox_test
\end{verbatim}

\begin{Shaded}
\begin{Highlighting}[]
\FunctionTok{library}\NormalTok{(dotwhisker)}
\FunctionTok{library}\NormalTok{(dplyr)}
\FunctionTok{library}\NormalTok{(lme4)}
\end{Highlighting}
\end{Shaded}

\begin{verbatim}
## Loading required package: Matrix
## 
## Attaching package: 'Matrix'
## 
## The following objects are masked from 'package:tidyr':
## 
##     expand, pack, unpack
\end{verbatim}

\begin{Shaded}
\begin{Highlighting}[]
\FunctionTok{library}\NormalTok{(emmeans)}
\end{Highlighting}
\end{Shaded}

\hypertarget{step-1-load-in-the-data-and-normalize-data}{%
\subsubsection{STEP 1 load in the data and normalize
data}\label{step-1-load-in-the-data-and-normalize-data}}

\begin{Shaded}
\begin{Highlighting}[]
\CommentTok{\#import Metadata sheet that is stored on Google Sheet }
\NormalTok{percent\_difference }\OtherTok{\textless{}{-}} \FunctionTok{read\_csv}\NormalTok{(}\StringTok{"\textasciitilde{}/GitHub/rc\_sfa{-}rc{-}3{-}wenas{-}modeling/Lit\_Review\_Fig/Output\_for\_analysis/analyte\_table\_DOC\_no3.csv"}\NormalTok{)}
\end{Highlighting}
\end{Shaded}

\begin{verbatim}
## Rows: 79 Columns: 23
## -- Column specification --------------------------------------------------------
## Delimiter: ","
## chr  (5): response_var, Koppen, TSF, Climate, Site
## dbl (18): Study_ID, Ne, Nc, Xe1, Xe2, Xe3, Xe4, Xe5, Xe6, Xe7, Xe8, Xe9, Xe1...
## 
## i Use `spec()` to retrieve the full column specification for this data.
## i Specify the column types or set `show_col_types = FALSE` to quiet this message.
\end{verbatim}

\begin{Shaded}
\begin{Highlighting}[]
\CommentTok{\# log transform the percentages }
\NormalTok{percent\_difference }\OtherTok{\textless{}{-}}\NormalTok{ percent\_difference }\SpecialCharTok{\%\textgreater{}\%} 
  \FunctionTok{mutate}\NormalTok{(}\AttributeTok{log\_pd =} \FunctionTok{log}\NormalTok{(}\FunctionTok{abs}\NormalTok{(Percent\_Difference))) }\CommentTok{\# should I be taking the absolute value of the negative percent\_differences?}

\NormalTok{percent\_difference\_DOC }\OtherTok{\textless{}{-}}\NormalTok{ percent\_difference }\SpecialCharTok{\%\textgreater{}\%} 
  \FunctionTok{filter}\NormalTok{(response\_var }\SpecialCharTok{==} \StringTok{"DOC\_mg\_C\_L"}\NormalTok{)}

\NormalTok{percent\_difference\_NO3 }\OtherTok{\textless{}{-}}\NormalTok{ percent\_difference }\SpecialCharTok{\%\textgreater{}\%} 
  \FunctionTok{filter}\NormalTok{(response\_var }\SpecialCharTok{==} \StringTok{"NO3\_mg\_per\_L\_as\_Nitrate"}\NormalTok{)}
\end{Highlighting}
\end{Shaded}

\hypertarget{step-2-check-normality-for-doc}{%
\subsubsection{STEP 2 check normality for
DOC}\label{step-2-check-normality-for-doc}}

\begin{Shaded}
\begin{Highlighting}[]
\FunctionTok{summary}\NormalTok{(percent\_difference\_DOC}\SpecialCharTok{$}\NormalTok{Percent\_Difference)}
\end{Highlighting}
\end{Shaded}

\begin{verbatim}
##    Min. 1st Qu.  Median    Mean 3rd Qu.    Max. 
##  -60.75   12.22   33.61   39.06   70.08  119.78
\end{verbatim}

\begin{Shaded}
\begin{Highlighting}[]
\CommentTok{\#Visualize data}
\FunctionTok{ggboxplot}\NormalTok{(percent\_difference\_DOC}\SpecialCharTok{$}\NormalTok{Percent\_Difference, }
          \AttributeTok{ylab =} \StringTok{"Percent Difference (\%)"}\NormalTok{, }\AttributeTok{xlab =} \ConstantTok{FALSE}\NormalTok{,}
          \AttributeTok{ggtheme =} \FunctionTok{theme\_minimal}\NormalTok{())}
\end{Highlighting}
\end{Shaded}

\includegraphics{04_Wenas_meta_stats_files/figure-latex/DOC-1.pdf}

\begin{Shaded}
\begin{Highlighting}[]
\CommentTok{\#perform shapiro{-}wilk test}
\FunctionTok{shapiro.test}\NormalTok{(percent\_difference\_DOC}\SpecialCharTok{$}\NormalTok{Percent\_Difference)}
\end{Highlighting}
\end{Shaded}

\begin{verbatim}
## 
##  Shapiro-Wilk normality test
## 
## data:  percent_difference_DOC$Percent_Difference
## W = 0.98185, p-value = 0.8513
\end{verbatim}

\begin{Shaded}
\begin{Highlighting}[]
\CommentTok{\# checking to see if the data is normal }
\FunctionTok{ggqqplot}\NormalTok{(percent\_difference\_DOC}\SpecialCharTok{$}\NormalTok{Percent\_Difference, }\AttributeTok{ylab =} \StringTok{"Percent Difference"}\NormalTok{,}
         \AttributeTok{ggtheme =} \FunctionTok{theme\_minimal}\NormalTok{())}
\end{Highlighting}
\end{Shaded}

\includegraphics{04_Wenas_meta_stats_files/figure-latex/DOC-2.pdf}

\begin{Shaded}
\begin{Highlighting}[]
\CommentTok{\# p{-}value is greater than 0.05 which indicates that the data is normally distributed}
\end{Highlighting}
\end{Shaded}

The percent differences for DOC are already normally distributed and do
NOT need to be transformed.

\hypertarget{step-3-perform-t-test-for-doc}{%
\subsubsection{STEP 3 perform t-test for
DOC}\label{step-3-perform-t-test-for-doc}}

\begin{Shaded}
\begin{Highlighting}[]
\CommentTok{\# }
\NormalTok{res }\OtherTok{\textless{}{-}} \FunctionTok{t.test}\NormalTok{(percent\_difference\_DOC}\SpecialCharTok{$}\NormalTok{Percent\_Difference, }\AttributeTok{mu =} \DecValTok{0}\NormalTok{)}
\CommentTok{\# printing the results}
\NormalTok{res}
\end{Highlighting}
\end{Shaded}

\begin{verbatim}
## 
##  One Sample t-test
## 
## data:  percent_difference_DOC$Percent_Difference
## t = 4.884, df = 31, p-value = 2.994e-05
## alternative hypothesis: true mean is not equal to 0
## 95 percent confidence interval:
##  22.74849 55.36954
## sample estimates:
## mean of x 
##  39.05901
\end{verbatim}

t is the test-statistic value (t=4.0412) df is the degrees of freedom
(df = 30) p-value is the significance level of the t-test (p-value =
0.0003409) conf.int is the confidence interval of the mean at 95\%
({[}17.14,52.16{]}) sample estimates is the mean value of the sample
(mean = 34.65)

The p-value of the test is 0.0003409, which is less than the
significance level alpha = 0.05. We can conclude that the mean percent
difference for DOC is significantly different from 0\%.

\hypertarget{step-4-do-the-same-thing-for-nitrate}{%
\subsubsection{STEP 4 do the same thing for
nitrate}\label{step-4-do-the-same-thing-for-nitrate}}

\begin{Shaded}
\begin{Highlighting}[]
\DocumentationTok{\#\#\#  check normality for NO3 }\AlertTok{\#\#\#}
\FunctionTok{summary}\NormalTok{(percent\_difference\_NO3}\SpecialCharTok{$}\NormalTok{Percent\_Difference)}
\end{Highlighting}
\end{Shaded}

\begin{verbatim}
##    Min. 1st Qu.  Median    Mean 3rd Qu.    Max. 
##  -94.74   37.41  160.00  183.26  308.24  807.51
\end{verbatim}

\begin{Shaded}
\begin{Highlighting}[]
\CommentTok{\#Visualize data}
\FunctionTok{ggboxplot}\NormalTok{(percent\_difference\_NO3}\SpecialCharTok{$}\NormalTok{Percent\_Difference, }
          \AttributeTok{ylab =} \StringTok{"Percent Difference (\%)"}\NormalTok{, }\AttributeTok{xlab =} \ConstantTok{FALSE}\NormalTok{,}
          \AttributeTok{ggtheme =} \FunctionTok{theme\_minimal}\NormalTok{())}
\end{Highlighting}
\end{Shaded}

\includegraphics{04_Wenas_meta_stats_files/figure-latex/NO3-1.pdf}

\begin{Shaded}
\begin{Highlighting}[]
\CommentTok{\#perform shapiro{-}wilk test}
\FunctionTok{shapiro.test}\NormalTok{(percent\_difference\_NO3}\SpecialCharTok{$}\NormalTok{Percent\_Difference)}
\end{Highlighting}
\end{Shaded}

\begin{verbatim}
## 
##  Shapiro-Wilk normality test
## 
## data:  percent_difference_NO3$Percent_Difference
## W = 0.92129, p-value = 0.003705
\end{verbatim}

\begin{Shaded}
\begin{Highlighting}[]
\CommentTok{\# p{-}value is less than 0.05 which indicates that the data is normally distributed}


\CommentTok{\# checking to see if the data is normal }
\FunctionTok{ggqqplot}\NormalTok{(percent\_difference\_NO3}\SpecialCharTok{$}\NormalTok{Percent\_Difference, }\AttributeTok{ylab =} \StringTok{"Percent Difference"}\NormalTok{,}
         \AttributeTok{ggtheme =} \FunctionTok{theme\_minimal}\NormalTok{())}
\end{Highlighting}
\end{Shaded}

\includegraphics{04_Wenas_meta_stats_files/figure-latex/NO3-2.pdf}

The percent differences for nitrate do not appear to be normally
distributed so lets do a log transform

\begin{Shaded}
\begin{Highlighting}[]
\CommentTok{\#perform shapiro{-}wilk test}
\FunctionTok{shapiro.test}\NormalTok{(percent\_difference\_NO3}\SpecialCharTok{$}\NormalTok{log\_pd)}
\end{Highlighting}
\end{Shaded}

\begin{verbatim}
## 
##  Shapiro-Wilk normality test
## 
## data:  percent_difference_NO3$log_pd
## W = NaN, p-value = NA
\end{verbatim}

\begin{Shaded}
\begin{Highlighting}[]
\CommentTok{\# p{-}value is less than 0.05 which indicates that the data is normally distributed}


\CommentTok{\# checking to see if the data is normal }
\FunctionTok{ggqqplot}\NormalTok{(percent\_difference\_NO3}\SpecialCharTok{$}\NormalTok{log\_pd, }\AttributeTok{ylab =} \StringTok{"Percent Difference"}\NormalTok{,}
         \AttributeTok{ggtheme =} \FunctionTok{theme\_minimal}\NormalTok{())}
\end{Highlighting}
\end{Shaded}

\begin{verbatim}
## Warning: Removed 2 rows containing non-finite values (`stat_qq()`).
\end{verbatim}

\begin{verbatim}
## Warning: Removed 2 rows containing non-finite values (`stat_qq_line()`).
## Removed 2 rows containing non-finite values (`stat_qq_line()`).
\end{verbatim}

\includegraphics{04_Wenas_meta_stats_files/figure-latex/log transform for NO3-1.pdf}
I don't think this looks better than the non-log transformed. What
should we go with here?

\begin{Shaded}
\begin{Highlighting}[]
\CommentTok{\# }
\NormalTok{res }\OtherTok{\textless{}{-}} \FunctionTok{t.test}\NormalTok{(percent\_difference\_NO3}\SpecialCharTok{$}\NormalTok{Percent\_Difference, }\AttributeTok{mu =} \DecValTok{0}\NormalTok{)}
\CommentTok{\# printing the results}
\NormalTok{res}
\end{Highlighting}
\end{Shaded}

\begin{verbatim}
## 
##  One Sample t-test
## 
## data:  percent_difference_NO3$Percent_Difference
## t = 6.6339, df = 46, p-value = 3.232e-08
## alternative hypothesis: true mean is not equal to 0
## 95 percent confidence interval:
##  127.6516 238.8619
## sample estimates:
## mean of x 
##  183.2567
\end{verbatim}

t is the test-statistic value (t = 6.5941) df is the degrees of freedom
(df = 47) p-value is the significance level of the t-test (p-value =
3.378e-08) conf.int is the confidence interval of the mean at 95\%
({[}124.93,234.63{]}) sample estimates is the mean value of the sample
(mean = 179.7861)

The p-value of the test is 3.378e, which is less than the significance
level alpha = 0.05. We can conclude that the mean percent difference for
NO3 is significantly different from 0\%.

\hypertarget{step-5-do-cis-of-effect-size-overlap-with-0}{%
\paragraph{STEP 5 Do CIs of effect size overlap with
0}\label{step-5-do-cis-of-effect-size-overlap-with-0}}

This will tell us if the effect sizes are significant or not

\begin{Shaded}
\begin{Highlighting}[]
\CommentTok{\# rm(list=ls(all=T)) \#this clears your Environment}


\FunctionTok{library}\NormalTok{(tidyverse)}
\FunctionTok{library}\NormalTok{(rstatix)}
\FunctionTok{library}\NormalTok{(ggpubr)}
\FunctionTok{library}\NormalTok{(ggplot2)}
\FunctionTok{library}\NormalTok{(coin)}
\FunctionTok{library}\NormalTok{(dotwhisker)}

\NormalTok{effect\_size }\OtherTok{\textless{}{-}} \FunctionTok{read\_csv}\NormalTok{(}\StringTok{"\textasciitilde{}/GitHub/rc\_sfa{-}rc{-}3{-}wenas{-}modeling/Lit\_Review\_Fig/Output\_for\_analysis/analyte\_table\_effect\_DOC\_no3.csv"}\NormalTok{)}
\end{Highlighting}
\end{Shaded}

\begin{verbatim}
## Rows: 80 Columns: 23
## -- Column specification --------------------------------------------------------
## Delimiter: ","
## chr  (5): response_var, Koppen, TSF, Climate, Site
## dbl (18): Study_ID, Ne, Nc, Xe1, Xe2, Xe3, Xe4, Xe5, Xe6, Xe7, Xe8, Xe9, Xe1...
## 
## i Use `spec()` to retrieve the full column specification for this data.
## i Specify the column types or set `show_col_types = FALSE` to quiet this message.
\end{verbatim}

\begin{Shaded}
\begin{Highlighting}[]
\NormalTok{effect\_size\_DOC }\OtherTok{\textless{}{-}}\NormalTok{ effect\_size }\SpecialCharTok{\%\textgreater{}\%} 
  \FunctionTok{filter}\NormalTok{(response\_var }\SpecialCharTok{==} \StringTok{"DOC\_mg\_C\_L"}\NormalTok{)}

\NormalTok{effect\_size\_NO3 }\OtherTok{\textless{}{-}}\NormalTok{ effect\_size }\SpecialCharTok{\%\textgreater{}\%} 
  \FunctionTok{filter}\NormalTok{(response\_var }\SpecialCharTok{==} \StringTok{"NO3\_mg\_per\_L\_as\_Nitrate"}\NormalTok{)}
\end{Highlighting}
\end{Shaded}

\hypertarget{i-want-to-see-the-relationship-between-the-effect-size-and-percent-change-that-peter-looked-at-really-quickly-with-the-dove-paper}{%
\subsubsection{I want to see the relationship between the effect size
and percent change that Peter looked at really quickly with the Dove
paper}\label{i-want-to-see-the-relationship-between-the-effect-size-and-percent-change-that-peter-looked-at-really-quickly-with-the-dove-paper}}

\begin{Shaded}
\begin{Highlighting}[]
\CommentTok{\# relationship between effect size and percent change }
\NormalTok{percent\_difference }\SpecialCharTok{\%\textgreater{}\%} 
  \FunctionTok{count}\NormalTok{(Study\_ID)}
\end{Highlighting}
\end{Shaded}

\begin{verbatim}
## # A tibble: 16 x 2
##    Study_ID     n
##       <dbl> <int>
##  1        1     4
##  2        2     3
##  3        3     1
##  4        4     2
##  5        5     4
##  6        9     2
##  7       10     2
##  8       11    18
##  9       12     5
## 10       15     8
## 11       16     6
## 12       17     3
## 13       21     4
## 14       22     6
## 15       26     2
## 16       27     9
\end{verbatim}

\begin{Shaded}
\begin{Highlighting}[]
\NormalTok{effect\_size }\SpecialCharTok{\%\textgreater{}\%} 
  \FunctionTok{count}\NormalTok{(Study\_ID)}
\end{Highlighting}
\end{Shaded}

\begin{verbatim}
## # A tibble: 16 x 2
##    Study_ID     n
##       <dbl> <int>
##  1        1     4
##  2        2     3
##  3        3     1
##  4        4     2
##  5        5     4
##  6        9     2
##  7       10     2
##  8       11    18
##  9       12     5
## 10       15     8
## 11       16     7
## 12       17     3
## 13       21     4
## 14       22     6
## 15       26     2
## 16       27     9
\end{verbatim}

\begin{Shaded}
\begin{Highlighting}[]
\NormalTok{es }\OtherTok{\textless{}{-}}\NormalTok{ effect\_size }\SpecialCharTok{\%\textgreater{}\%} 
  \FunctionTok{filter}\NormalTok{(}\SpecialCharTok{!}\FunctionTok{row\_number}\NormalTok{() }\SpecialCharTok{\%in\%} \FunctionTok{c}\NormalTok{(}\DecValTok{37}\NormalTok{))}
  
\NormalTok{pd\_es }\OtherTok{\textless{}{-}}\NormalTok{ percent\_difference }\SpecialCharTok{\%\textgreater{}\%} 
  \FunctionTok{select}\NormalTok{(}\StringTok{"Study\_ID"}\NormalTok{, }\StringTok{"response\_var"}\NormalTok{, }\StringTok{"Percent\_Difference"}\NormalTok{) }\SpecialCharTok{\%\textgreater{}\%} 
  \FunctionTok{mutate}\NormalTok{(}\AttributeTok{Effect\_Size =} \FunctionTok{paste}\NormalTok{(es}\SpecialCharTok{$}\NormalTok{Effect\_size))}

\FunctionTok{ggplot}\NormalTok{(pd\_es) }\SpecialCharTok{+}
  \FunctionTok{geom\_point}\NormalTok{(}\FunctionTok{aes}\NormalTok{(}\AttributeTok{x =}\NormalTok{ Percent\_Difference, }\AttributeTok{y =} \FunctionTok{as.numeric}\NormalTok{(Effect\_Size), }\AttributeTok{color =} \StringTok{"red"}\NormalTok{)) }\SpecialCharTok{+}
  \FunctionTok{facet\_wrap}\NormalTok{(}\SpecialCharTok{\textasciitilde{}}\NormalTok{response\_var) }\SpecialCharTok{+}
  \FunctionTok{theme\_bw}\NormalTok{()}
\end{Highlighting}
\end{Shaded}

\includegraphics{04_Wenas_meta_stats_files/figure-latex/unnamed-chunk-2-1.pdf}
The equation for percent change in Dove's paper is D = 100(1-e\^{}ln(R))

This looks like a log relationship so is there where they get that
equation for the relationship between effect size and percent
change\ldots{}

\hypertarget{calculating-ci}{%
\subsubsection{CALCULATING CI}\label{calculating-ci}}

\begin{Shaded}
\begin{Highlighting}[]
\CommentTok{\# Calculating a confidence interval for all of the effect sizes}
\CommentTok{\# calculating the mean effect\_size by climate}
\NormalTok{climate\_summary\_effect }\OtherTok{\textless{}{-}}\NormalTok{ effect\_size }\SpecialCharTok{\%\textgreater{}\%}
  \FunctionTok{group\_by}\NormalTok{(Koppen, response\_var, Climate) }\SpecialCharTok{\%\textgreater{}\%}
  \FunctionTok{summarise}\NormalTok{(}\AttributeTok{sd\_effect =} \FunctionTok{sd}\NormalTok{(Effect\_size, }\AttributeTok{na.rm =} \ConstantTok{TRUE}\NormalTok{),}
            \AttributeTok{n\_effect =} \FunctionTok{n}\NormalTok{(),}
            \AttributeTok{Effect\_size =} \FunctionTok{mean}\NormalTok{(Effect\_size, }\AttributeTok{na.rm =} \ConstantTok{TRUE}\NormalTok{)) }\SpecialCharTok{\%\textgreater{}\%} 
 \FunctionTok{mutate}\NormalTok{(}\AttributeTok{se\_effect =}\NormalTok{ sd\_effect }\SpecialCharTok{/} \FunctionTok{sqrt}\NormalTok{(n\_effect),}
         \AttributeTok{lower\_ci\_effect =}\NormalTok{ Effect\_size }\SpecialCharTok{{-}} \FunctionTok{qt}\NormalTok{(}\DecValTok{1} \SpecialCharTok{{-}}\NormalTok{ (}\FloatTok{0.05} \SpecialCharTok{/} \DecValTok{2}\NormalTok{), n\_effect }\SpecialCharTok{{-}} \DecValTok{1}\NormalTok{) }\SpecialCharTok{*}\NormalTok{ se\_effect,}
         \AttributeTok{upper\_ci\_effect =}\NormalTok{ Effect\_size }\SpecialCharTok{+} \FunctionTok{qt}\NormalTok{(}\DecValTok{1} \SpecialCharTok{{-}}\NormalTok{ (}\FloatTok{0.05} \SpecialCharTok{/} \DecValTok{2}\NormalTok{), n\_effect }\SpecialCharTok{{-}} \DecValTok{1}\NormalTok{) }\SpecialCharTok{*}\NormalTok{ se\_effect)}
\end{Highlighting}
\end{Shaded}

\begin{verbatim}
## `summarise()` has grouped output by 'Koppen', 'response_var'. You can override
## using the `.groups` argument.
\end{verbatim}

\begin{Shaded}
\begin{Highlighting}[]
\CommentTok{\# Calculating a confidence interval for all of the effect sizes}
\CommentTok{\# calculating the mean effect\_size by climate}
\NormalTok{TSF\_summary\_effect }\OtherTok{\textless{}{-}}\NormalTok{ effect\_size }\SpecialCharTok{\%\textgreater{}\%}
  \FunctionTok{group\_by}\NormalTok{(TSF, response\_var) }\SpecialCharTok{\%\textgreater{}\%}
  \FunctionTok{summarise}\NormalTok{(}\AttributeTok{sd\_effect =} \FunctionTok{sd}\NormalTok{(Effect\_size, }\AttributeTok{na.rm =} \ConstantTok{TRUE}\NormalTok{),}
            \AttributeTok{n\_effect =} \FunctionTok{n}\NormalTok{(),}
            \AttributeTok{Effect\_size =} \FunctionTok{mean}\NormalTok{(Effect\_size, }\AttributeTok{na.rm =} \ConstantTok{TRUE}\NormalTok{)) }\SpecialCharTok{\%\textgreater{}\%} 
 \FunctionTok{mutate}\NormalTok{(}\AttributeTok{se\_effect =}\NormalTok{ sd\_effect }\SpecialCharTok{/} \FunctionTok{sqrt}\NormalTok{(n\_effect),}
         \AttributeTok{lower\_ci\_effect =}\NormalTok{ Effect\_size }\SpecialCharTok{{-}} \FunctionTok{qt}\NormalTok{(}\DecValTok{1} \SpecialCharTok{{-}}\NormalTok{ (}\FloatTok{0.05} \SpecialCharTok{/} \DecValTok{2}\NormalTok{), n\_effect }\SpecialCharTok{{-}} \DecValTok{1}\NormalTok{) }\SpecialCharTok{*}\NormalTok{ se\_effect,}
         \AttributeTok{upper\_ci\_effect =}\NormalTok{ Effect\_size }\SpecialCharTok{+} \FunctionTok{qt}\NormalTok{(}\DecValTok{1} \SpecialCharTok{{-}}\NormalTok{ (}\FloatTok{0.05} \SpecialCharTok{/} \DecValTok{2}\NormalTok{), n\_effect }\SpecialCharTok{{-}} \DecValTok{1}\NormalTok{) }\SpecialCharTok{*}\NormalTok{ se\_effect)}
\end{Highlighting}
\end{Shaded}

\begin{verbatim}
## `summarise()` has grouped output by 'TSF'. You can override using the `.groups`
## argument.
\end{verbatim}

\begin{verbatim}
## Warning: There were 2 warnings in `mutate()`.
## The first warning was:
## i In argument: `lower_ci_effect = Effect_size - qt(1 - (0.05/2), n_effect - 1)
##   * se_effect`.
## i In group 4: `TSF = ">10 years"`.
## Caused by warning in `qt()`:
## ! NaNs produced
## i Run `dplyr::last_dplyr_warnings()` to see the 1 remaining warning.
\end{verbatim}

\hypertarget{plots}{%
\subsubsection{PLOTS}\label{plots}}

\begin{Shaded}
\begin{Highlighting}[]
\DocumentationTok{\#\#\# Plotting }\AlertTok{\#\#\#}

\CommentTok{\# Ploting \#}
\NormalTok{vn }\OtherTok{=} \FunctionTok{expression}\NormalTok{(}\FunctionTok{paste}\NormalTok{(}\StringTok{""}\SpecialCharTok{*}\NormalTok{N}\SpecialCharTok{*}\NormalTok{O[}\DecValTok{3}\NormalTok{]}\SpecialCharTok{\^{}}\StringTok{"{-}"}\NormalTok{))}

\CommentTok{\# geom\_jitter for climate \# }
\FunctionTok{ggplot}\NormalTok{(effect\_size, }\FunctionTok{aes}\NormalTok{(Effect\_size, Koppen, }\AttributeTok{color =}\NormalTok{ response\_var),}
       \AttributeTok{position =} \FunctionTok{position\_dodge}\NormalTok{(}\AttributeTok{width =} \SpecialCharTok{{-}}\FloatTok{0.5}\NormalTok{)) }\SpecialCharTok{+}
  \FunctionTok{geom\_jitter}\NormalTok{(}\AttributeTok{position =} \FunctionTok{position\_jitter}\NormalTok{(}\FloatTok{0.1}\NormalTok{), }\AttributeTok{alpha =} \FloatTok{0.4}\NormalTok{, }\AttributeTok{size =} \DecValTok{1}\NormalTok{) }\SpecialCharTok{+}
  \FunctionTok{geom\_pointrange}\NormalTok{(}\FunctionTok{aes}\NormalTok{(}\AttributeTok{xmin =}\NormalTok{ lower\_ci\_effect, }\AttributeTok{xmax =}\NormalTok{ upper\_ci\_effect, }
                      \AttributeTok{color =}\NormalTok{ response\_var),}
                      \AttributeTok{position =} \FunctionTok{position\_dodge}\NormalTok{(}\AttributeTok{width =} \SpecialCharTok{{-}}\FloatTok{0.5}\NormalTok{), }\AttributeTok{size =} \FloatTok{1.0}\NormalTok{, }\AttributeTok{data =}\NormalTok{ climate\_summary\_effect) }\SpecialCharTok{+}
  \FunctionTok{geom\_vline}\NormalTok{(}\AttributeTok{xintercept =} \DecValTok{0}\NormalTok{, }\AttributeTok{linewidth =} \FloatTok{0.5}\NormalTok{, }\AttributeTok{color =} \StringTok{"red"}\NormalTok{) }\SpecialCharTok{+}
  \FunctionTok{ylab}\NormalTok{(}\StringTok{"Koppen (Climate Classification)"}\NormalTok{) }\SpecialCharTok{+}
  \FunctionTok{xlab}\NormalTok{(}\StringTok{"Effect Size (ln[R])"}\NormalTok{) }\SpecialCharTok{+}
  \FunctionTok{ggtitle}\NormalTok{(}\StringTok{"95\% CI of Effect Size"}\NormalTok{) }\SpecialCharTok{+}
  \FunctionTok{geom\_segment}\NormalTok{(}\FunctionTok{aes}\NormalTok{(}\AttributeTok{x =} \FloatTok{5.5}\NormalTok{, }\AttributeTok{xend =} \FloatTok{5.5}\NormalTok{, }\AttributeTok{y =} \FloatTok{6.5}\NormalTok{, }\AttributeTok{yend =} \FloatTok{4.5}\NormalTok{), }\AttributeTok{color =} \StringTok{"black"}\NormalTok{) }\SpecialCharTok{+}
  \FunctionTok{geom\_segment}\NormalTok{(}\FunctionTok{aes}\NormalTok{(}\AttributeTok{x =} \FloatTok{4.75}\NormalTok{, }\AttributeTok{xend =} \FloatTok{5.5}\NormalTok{, }\AttributeTok{y =} \FloatTok{6.5}\NormalTok{, }\AttributeTok{yend =} \FloatTok{6.5}\NormalTok{), }\AttributeTok{color =} \StringTok{"black"}\NormalTok{) }\SpecialCharTok{+}
  \FunctionTok{geom\_segment}\NormalTok{(}\FunctionTok{aes}\NormalTok{(}\AttributeTok{x =} \FloatTok{4.75}\NormalTok{, }\AttributeTok{xend =} \FloatTok{5.5}\NormalTok{, }\AttributeTok{y =} \FloatTok{4.5}\NormalTok{, }\AttributeTok{yend =} \FloatTok{4.5}\NormalTok{), }\AttributeTok{color =} \StringTok{"black"}\NormalTok{) }\SpecialCharTok{+}
  
  \FunctionTok{geom\_segment}\NormalTok{(}\FunctionTok{aes}\NormalTok{(}\AttributeTok{x =} \FloatTok{5.5}\NormalTok{, }\AttributeTok{xend =} \FloatTok{5.5}\NormalTok{, }\AttributeTok{y =} \FloatTok{4.3}\NormalTok{, }\AttributeTok{yend =} \FloatTok{1.5}\NormalTok{), }\AttributeTok{color =} \StringTok{"black"}\NormalTok{) }\SpecialCharTok{+}
  \FunctionTok{geom\_segment}\NormalTok{(}\FunctionTok{aes}\NormalTok{(}\AttributeTok{x =} \FloatTok{4.75}\NormalTok{, }\AttributeTok{xend =} \FloatTok{5.5}\NormalTok{, }\AttributeTok{y =} \FloatTok{4.3}\NormalTok{, }\AttributeTok{yend =} \FloatTok{4.3}\NormalTok{), }\AttributeTok{color =} \StringTok{"black"}\NormalTok{) }\SpecialCharTok{+}
  \FunctionTok{geom\_segment}\NormalTok{(}\FunctionTok{aes}\NormalTok{(}\AttributeTok{x =} \FloatTok{4.75}\NormalTok{, }\AttributeTok{xend =} \FloatTok{5.5}\NormalTok{, }\AttributeTok{y =} \FloatTok{1.5}\NormalTok{, }\AttributeTok{yend =} \FloatTok{1.5}\NormalTok{), }\AttributeTok{color =} \StringTok{"black"}\NormalTok{) }\SpecialCharTok{+}
   
  \FunctionTok{geom\_segment}\NormalTok{(}\FunctionTok{aes}\NormalTok{(}\AttributeTok{x =} \FloatTok{5.5}\NormalTok{, }\AttributeTok{xend =} \FloatTok{5.5}\NormalTok{, }\AttributeTok{y =} \FloatTok{1.3}\NormalTok{, }\AttributeTok{yend =} \FloatTok{0.5}\NormalTok{), }\AttributeTok{color =} \StringTok{"black"}\NormalTok{) }\SpecialCharTok{+}
  \FunctionTok{geom\_segment}\NormalTok{(}\FunctionTok{aes}\NormalTok{(}\AttributeTok{x =} \FloatTok{4.75}\NormalTok{, }\AttributeTok{xend =} \FloatTok{5.5}\NormalTok{, }\AttributeTok{y =} \FloatTok{1.3}\NormalTok{, }\AttributeTok{yend =} \FloatTok{1.3}\NormalTok{), }\AttributeTok{color =} \StringTok{"black"}\NormalTok{) }\SpecialCharTok{+}
  \FunctionTok{geom\_segment}\NormalTok{(}\FunctionTok{aes}\NormalTok{(}\AttributeTok{x =} \FloatTok{4.75}\NormalTok{, }\AttributeTok{xend =} \FloatTok{5.5}\NormalTok{, }\AttributeTok{y =} \FloatTok{0.5}\NormalTok{, }\AttributeTok{yend =} \FloatTok{0.5}\NormalTok{), }\AttributeTok{color =} \StringTok{"black"}\NormalTok{) }\SpecialCharTok{+}
  \FunctionTok{annotate}\NormalTok{(}\StringTok{"text"}\NormalTok{,}
           \AttributeTok{x =} \FunctionTok{c}\NormalTok{(}\FloatTok{6.1}\NormalTok{, }\FloatTok{6.1}\NormalTok{, }\FloatTok{6.1}\NormalTok{),}
           \AttributeTok{y =} \FunctionTok{c}\NormalTok{(}\FloatTok{5.5}\NormalTok{, }\DecValTok{3}\NormalTok{, }\DecValTok{1}\NormalTok{),}
           \AttributeTok{label =} \FunctionTok{c}\NormalTok{(}\StringTok{"Cold"}\NormalTok{, }\StringTok{"Temperate"}\NormalTok{, }\StringTok{"Arid"}\NormalTok{),}
           \AttributeTok{family =} \StringTok{""}\NormalTok{, }\AttributeTok{fontface =} \StringTok{"bold"}\NormalTok{) }\SpecialCharTok{+}
  \FunctionTok{scale\_color\_manual}\NormalTok{(}\AttributeTok{values =} \FunctionTok{c}\NormalTok{(}\StringTok{"\#00AFBB"}\NormalTok{, }\StringTok{"\#E7B800"}\NormalTok{),}
                     \AttributeTok{guide =} \FunctionTok{guide\_legend}\NormalTok{(}\AttributeTok{title =} \StringTok{"Analyte"}\NormalTok{),}
                     \AttributeTok{labels =} \FunctionTok{c}\NormalTok{(}\StringTok{\textquotesingle{}DOC\textquotesingle{}}\NormalTok{, vn)) }\SpecialCharTok{+}
  \FunctionTok{theme\_classic}\NormalTok{() }\SpecialCharTok{+}
  \FunctionTok{theme}\NormalTok{(}\AttributeTok{axis.text.x =} \FunctionTok{element\_text}\NormalTok{(}\AttributeTok{size =} \DecValTok{15}\NormalTok{),}
        \AttributeTok{axis.text.y =} \FunctionTok{element\_text}\NormalTok{(}\AttributeTok{size =} \DecValTok{15}\NormalTok{),}
        \AttributeTok{axis.title.x =} \FunctionTok{element\_text}\NormalTok{(}\AttributeTok{size =} \DecValTok{20}\NormalTok{),}
        \AttributeTok{axis.title.y =} \FunctionTok{element\_text}\NormalTok{(}\AttributeTok{size =} \DecValTok{20}\NormalTok{),}
        \AttributeTok{legend.text =} \FunctionTok{element\_text}\NormalTok{(}\AttributeTok{size =} \DecValTok{15}\NormalTok{),}
        \AttributeTok{legend.position =} \FunctionTok{c}\NormalTok{(}\FloatTok{0.1}\NormalTok{, }\FloatTok{0.15}\NormalTok{),}
        \AttributeTok{title =} \FunctionTok{element\_text}\NormalTok{(}\AttributeTok{size =} \DecValTok{10}\NormalTok{))}
\end{Highlighting}
\end{Shaded}

\begin{verbatim}
## Warning: `position_dodge()` requires non-overlapping x intervals
\end{verbatim}

\includegraphics{04_Wenas_meta_stats_files/figure-latex/unnamed-chunk-4-1.pdf}

\begin{Shaded}
\begin{Highlighting}[]
\FunctionTok{ggsave}\NormalTok{(}\StringTok{"site.effect.geom\_pointrange.climate.pdf"}\NormalTok{,}
       \AttributeTok{path =}\NormalTok{ (}\StringTok{"\textasciitilde{}/GitHub/rc\_sfa{-}rc{-}3{-}wenas{-}modeling/Lit\_Review\_Fig/initial\_plots/03\_Wenas\_effect\_size"}\NormalTok{),}
       \AttributeTok{width =} \DecValTok{10}\NormalTok{, }\AttributeTok{height =} \DecValTok{8}\NormalTok{, }\AttributeTok{units =} \StringTok{"in"}\NormalTok{)}
\end{Highlighting}
\end{Shaded}

\begin{verbatim}
## Warning: `position_dodge()` requires non-overlapping x intervals
\end{verbatim}

\begin{Shaded}
\begin{Highlighting}[]
\CommentTok{\# geom\_jitter for TSF \# }
\CommentTok{\# Creating an order in which I want to plot the y{-}axis}
\NormalTok{level\_order }\OtherTok{\textless{}{-}} \FunctionTok{c}\NormalTok{(}\StringTok{\textquotesingle{}0{-}1 years\textquotesingle{}}\NormalTok{, }\StringTok{\textquotesingle{}2{-}3 years\textquotesingle{}}\NormalTok{, }\StringTok{\textquotesingle{}\textgreater{}10 years\textquotesingle{}}\NormalTok{) }

\FunctionTok{ggplot}\NormalTok{(effect\_size, }\FunctionTok{aes}\NormalTok{(Effect\_size, TSF, }\AttributeTok{color =}\NormalTok{ response\_var)) }\SpecialCharTok{+}
  \FunctionTok{geom\_jitter}\NormalTok{(}\AttributeTok{position =} \FunctionTok{position\_jitter}\NormalTok{(}\FloatTok{0.2}\NormalTok{), }\AttributeTok{alpha =} \FloatTok{0.4}\NormalTok{, }\AttributeTok{size =} \DecValTok{1}\NormalTok{) }\SpecialCharTok{+}
  \FunctionTok{geom\_pointrange}\NormalTok{(}\FunctionTok{aes}\NormalTok{(}\AttributeTok{xmin =}\NormalTok{ lower\_ci\_effect, }\AttributeTok{xmax =}\NormalTok{ upper\_ci\_effect, }
                      \AttributeTok{color =}\NormalTok{ response\_var),}
                      \AttributeTok{position =} \FunctionTok{position\_dodge}\NormalTok{(}\AttributeTok{width =} \SpecialCharTok{{-}}\FloatTok{0.5}\NormalTok{), }\AttributeTok{size =} \FloatTok{1.0}\NormalTok{, }\AttributeTok{data =}\NormalTok{ TSF\_summary\_effect) }\SpecialCharTok{+}
  \FunctionTok{geom\_vline}\NormalTok{(}\AttributeTok{xintercept =} \DecValTok{0}\NormalTok{, }\AttributeTok{linewidth =} \FloatTok{0.5}\NormalTok{, }\AttributeTok{color =} \StringTok{"red"}\NormalTok{) }\SpecialCharTok{+}
  \FunctionTok{scale\_y\_discrete}\NormalTok{(}\AttributeTok{limit =}\NormalTok{ level\_order) }\SpecialCharTok{+}
  \FunctionTok{xlab}\NormalTok{(}\StringTok{"Effect Size (ln[R])"}\NormalTok{) }\SpecialCharTok{+}
  \FunctionTok{ggtitle}\NormalTok{(}\StringTok{"95\% CI of Effect Size"}\NormalTok{) }\SpecialCharTok{+}
  \FunctionTok{scale\_color\_manual}\NormalTok{(}\AttributeTok{values =} \FunctionTok{c}\NormalTok{(}\StringTok{"\#00AFBB"}\NormalTok{, }\StringTok{"\#E7B800"}\NormalTok{),}
                     \AttributeTok{guide =} \FunctionTok{guide\_legend}\NormalTok{(}\AttributeTok{title =} \StringTok{"Analyte"}\NormalTok{),}
                     \AttributeTok{labels =} \FunctionTok{c}\NormalTok{(}\StringTok{\textquotesingle{}DOC\textquotesingle{}}\NormalTok{, vn)) }\SpecialCharTok{+}
  \FunctionTok{theme\_classic}\NormalTok{() }\SpecialCharTok{+}
  \FunctionTok{theme}\NormalTok{(}\AttributeTok{axis.text.x =} \FunctionTok{element\_text}\NormalTok{(}\AttributeTok{size =} \DecValTok{15}\NormalTok{),}
        \AttributeTok{axis.text.y =} \FunctionTok{element\_text}\NormalTok{(}\AttributeTok{size =} \DecValTok{15}\NormalTok{),}
        \AttributeTok{axis.title.x =} \FunctionTok{element\_text}\NormalTok{(}\AttributeTok{size =} \DecValTok{20}\NormalTok{),}
        \AttributeTok{axis.title.y =} \FunctionTok{element\_text}\NormalTok{(}\AttributeTok{size =} \DecValTok{20}\NormalTok{),}
        \AttributeTok{legend.title =} \FunctionTok{element\_text}\NormalTok{(}\AttributeTok{size =} \DecValTok{20}\NormalTok{),}
        \AttributeTok{legend.text =} \FunctionTok{element\_text}\NormalTok{(}\AttributeTok{size =} \DecValTok{15}\NormalTok{),}
        \AttributeTok{legend.position =} \FunctionTok{c}\NormalTok{(}\FloatTok{0.2}\NormalTok{, }\FloatTok{0.15}\NormalTok{),}
        \AttributeTok{title =} \FunctionTok{element\_text}\NormalTok{(}\AttributeTok{size =} \DecValTok{10}\NormalTok{))}
\end{Highlighting}
\end{Shaded}

\begin{verbatim}
## Warning: `position_dodge()` requires non-overlapping x intervals
\end{verbatim}

\begin{verbatim}
## Warning: Removed 10 rows containing missing values (`geom_point()`).
\end{verbatim}

\begin{verbatim}
## Warning: Removed 2 rows containing missing values (`geom_pointrange()`).
\end{verbatim}

\begin{verbatim}
## Warning: Removed 1 rows containing missing values (`geom_segment()`).
\end{verbatim}

\includegraphics{04_Wenas_meta_stats_files/figure-latex/unnamed-chunk-4-2.pdf}

\begin{Shaded}
\begin{Highlighting}[]
\FunctionTok{ggsave}\NormalTok{(}\StringTok{"site.difference.geom\_pointrange.TSF.pdf"}\NormalTok{,}
       \AttributeTok{path =}\NormalTok{ (}\StringTok{"\textasciitilde{}/GitHub/rc\_sfa{-}rc{-}3{-}wenas{-}modeling/Lit\_Review\_Fig/initial\_plots/03\_Wenas\_effect\_size"}\NormalTok{),}
       \AttributeTok{width =} \DecValTok{10}\NormalTok{, }\AttributeTok{height =} \DecValTok{8}\NormalTok{, }\AttributeTok{units =} \StringTok{"in"}\NormalTok{)}
\end{Highlighting}
\end{Shaded}

\begin{verbatim}
## Warning: `position_dodge()` requires non-overlapping x intervals
\end{verbatim}

\begin{verbatim}
## Warning: Removed 10 rows containing missing values (`geom_point()`).
\end{verbatim}

\begin{verbatim}
## Warning: Removed 2 rows containing missing values (`geom_pointrange()`).
\end{verbatim}

\begin{verbatim}
## Warning: Removed 1 rows containing missing values (`geom_segment()`).
\end{verbatim}

\hypertarget{step-6-random-effects-model}{%
\subsubsection{STEP 6 random effects
model}\label{step-6-random-effects-model}}

\begin{Shaded}
\begin{Highlighting}[]
\NormalTok{mDOC\_climate }\OtherTok{\textless{}{-}} \FunctionTok{lmer}\NormalTok{(Effect\_size }\SpecialCharTok{\textasciitilde{}}\NormalTok{ Climate }\SpecialCharTok{+}\NormalTok{ (}\DecValTok{1}\SpecialCharTok{|}\NormalTok{Study\_ID), }\AttributeTok{data =}\NormalTok{ effect\_size\_DOC)}
\end{Highlighting}
\end{Shaded}

\begin{verbatim}
## boundary (singular) fit: see help('isSingular')
\end{verbatim}

\begin{Shaded}
\begin{Highlighting}[]
\FunctionTok{plot}\NormalTok{(mDOC\_climate)}
\end{Highlighting}
\end{Shaded}

\includegraphics{04_Wenas_meta_stats_files/figure-latex/DOC random effects-1.pdf}

\begin{Shaded}
\begin{Highlighting}[]
\FunctionTok{qqnorm}\NormalTok{(}\FunctionTok{resid}\NormalTok{(mDOC\_climate))}
\end{Highlighting}
\end{Shaded}

\includegraphics{04_Wenas_meta_stats_files/figure-latex/DOC random effects-2.pdf}

\begin{Shaded}
\begin{Highlighting}[]
\FunctionTok{anova}\NormalTok{(mDOC\_climate) }\CommentTok{\# f{-}value = 0.7755}
\end{Highlighting}
\end{Shaded}

\begin{verbatim}
## Analysis of Variance Table
##         npar  Sum Sq Mean Sq F value
## Climate    1 0.11483 0.11483  0.7755
\end{verbatim}

\begin{Shaded}
\begin{Highlighting}[]
\FunctionTok{summary}\NormalTok{(mDOC\_climate) }
\end{Highlighting}
\end{Shaded}

\begin{verbatim}
## Linear mixed model fit by REML ['lmerMod']
## Formula: Effect_size ~ Climate + (1 | Study_ID)
##    Data: effect_size_DOC
## 
## REML criterion at convergence: 33.4
## 
## Scaled residuals: 
##     Min      1Q  Median      3Q     Max 
## -2.9881 -0.5508  0.1148  0.7526  1.3276 
## 
## Random effects:
##  Groups   Name        Variance Std.Dev.
##  Study_ID (Intercept) 0.0000   0.0000  
##  Residual             0.1481   0.3848  
## Number of obs: 32, groups:  Study_ID, 9
## 
## Fixed effects:
##                  Estimate Std. Error t value
## (Intercept)        0.3354     0.1028   3.261
## ClimateTemperate  -0.1208     0.1371  -0.881
## 
## Correlation of Fixed Effects:
##             (Intr)
## ClimatTmprt -0.750
## optimizer (nloptwrap) convergence code: 0 (OK)
## boundary (singular) fit: see help('isSingular')
\end{verbatim}

Column standard deviation is a measure of how much variability in the
dependent measure there is due to the Study\_IDs (our random effects).
You can see that the Study\_ID is adding no additional variance. I think
this would suggest that we can use a random intercept model for DOC. The
random intercept framework (1\textbar Study\_ID) tells the model that it
should expect that there are going to be multiple responses per
Study\_ID and these responses will depend on the Study\_ID ``baseline''
This framework assumes that whatever the effect of effect\_size is, its
going to be the same for all Study\_IDs and considering our variance is
0 I think this is a safe assumption. If this assumption/interpretation
is wrong on my end than we can look at doing a random slopes structure
which would look like lmer(effect\_size \textasciitilde{} Climate +
(1+Climate\textbar Study\_ID))

``Residual'' which stands for the variability that's not due to
Study\_ID. This is our ``ε'' again, the ``random'' deviations from the
predicted values that are not due to Study\_ID.

\begin{Shaded}
\begin{Highlighting}[]
\CommentTok{\#emmeans(mDOC, list(pairwise \textasciitilde{} Study\_ID), adjust = "tukey")}

\CommentTok{\# fire}
\NormalTok{mDOC\_fire }\OtherTok{\textless{}{-}} \FunctionTok{lmer}\NormalTok{(Effect\_size }\SpecialCharTok{\textasciitilde{}}\NormalTok{ TSF }\SpecialCharTok{+}\NormalTok{ (}\DecValTok{1}\SpecialCharTok{|}\NormalTok{Study\_ID), }\AttributeTok{data =}\NormalTok{ effect\_size\_DOC)}

\FunctionTok{plot}\NormalTok{(mDOC\_fire)}
\end{Highlighting}
\end{Shaded}

\includegraphics{04_Wenas_meta_stats_files/figure-latex/DOC random effects fire-1.pdf}

\begin{Shaded}
\begin{Highlighting}[]
\FunctionTok{qqnorm}\NormalTok{(}\FunctionTok{resid}\NormalTok{(mDOC\_fire))}
\end{Highlighting}
\end{Shaded}

\includegraphics{04_Wenas_meta_stats_files/figure-latex/DOC random effects fire-2.pdf}

\begin{Shaded}
\begin{Highlighting}[]
\FunctionTok{anova}\NormalTok{(mDOC\_fire) }\CommentTok{\# f{-}value = 0.7755}
\end{Highlighting}
\end{Shaded}

\begin{verbatim}
## Analysis of Variance Table
##     npar  Sum Sq Mean Sq F value
## TSF    3 0.32857 0.10952  0.9201
\end{verbatim}

\begin{Shaded}
\begin{Highlighting}[]
\FunctionTok{summary}\NormalTok{(mDOC\_fire) }
\end{Highlighting}
\end{Shaded}

\begin{verbatim}
## Linear mixed model fit by REML ['lmerMod']
## Formula: Effect_size ~ TSF + (1 | Study_ID)
##    Data: effect_size_DOC
## 
## REML criterion at convergence: 34.5
## 
## Scaled residuals: 
##      Min       1Q   Median       3Q      Max 
## -2.43954 -0.48296 -0.01462  0.50374  1.57846 
## 
## Random effects:
##  Groups   Name        Variance Std.Dev.
##  Study_ID (Intercept) 0.07828  0.2798  
##  Residual             0.11904  0.3450  
## Number of obs: 32, groups:  Study_ID, 9
## 
## Fixed effects:
##              Estimate Std. Error t value
## (Intercept)  -0.09175    0.25156  -0.365
## TSF0-1 years  0.47015    0.29446   1.597
## TSF2-3 years  0.38161    0.31738   1.202
## TSF4-5 years  0.30777    0.31888   0.965
## 
## Correlation of Fixed Effects:
##             (Intr) TSF0-y TSF2-y
## TSF0-1years -0.854              
## TSF2-3years -0.793  0.813       
## TSF4-5years -0.789  0.783  0.766
\end{verbatim}

This is the model with TimeSinceFire rather than climate. It shows that
Study\_ID is still not adding a ton of variance. The T-value for 0-1
year is the highest meaning that its the most significant.

\begin{Shaded}
\begin{Highlighting}[]
\NormalTok{mNO3\_climate }\OtherTok{\textless{}{-}} \FunctionTok{lmer}\NormalTok{(Effect\_size }\SpecialCharTok{\textasciitilde{}}\NormalTok{ Climate }\SpecialCharTok{+}\NormalTok{ (}\DecValTok{1}\SpecialCharTok{|}\NormalTok{Study\_ID), }\AttributeTok{data =}\NormalTok{ effect\_size\_NO3)}

\FunctionTok{plot}\NormalTok{(mNO3\_climate)}
\end{Highlighting}
\end{Shaded}

\includegraphics{04_Wenas_meta_stats_files/figure-latex/NO3 random effects-1.pdf}

\begin{Shaded}
\begin{Highlighting}[]
\FunctionTok{qqnorm}\NormalTok{(}\FunctionTok{resid}\NormalTok{(mNO3\_climate))}
\end{Highlighting}
\end{Shaded}

\includegraphics{04_Wenas_meta_stats_files/figure-latex/NO3 random effects-2.pdf}

\begin{Shaded}
\begin{Highlighting}[]
\FunctionTok{anova}\NormalTok{(mNO3\_climate) }\CommentTok{\# f{-}value = 0.5874}
\end{Highlighting}
\end{Shaded}

\begin{verbatim}
## Analysis of Variance Table
##         npar  Sum Sq Mean Sq F value
## Climate    2 0.65126 0.32563  0.5874
\end{verbatim}

\begin{Shaded}
\begin{Highlighting}[]
\FunctionTok{summary}\NormalTok{(mNO3\_climate) }
\end{Highlighting}
\end{Shaded}

\begin{verbatim}
## Linear mixed model fit by REML ['lmerMod']
## Formula: Effect_size ~ Climate + (1 | Study_ID)
##    Data: effect_size_NO3
## 
## REML criterion at convergence: 118.6
## 
## Scaled residuals: 
##     Min      1Q  Median      3Q     Max 
## -3.6271 -0.4735 -0.1199  0.4802  2.8147 
## 
## Random effects:
##  Groups   Name        Variance Std.Dev.
##  Study_ID (Intercept) 0.2894   0.5379  
##  Residual             0.5543   0.7445  
## Number of obs: 48, groups:  Study_ID, 13
## 
## Fixed effects:
##                  Estimate Std. Error t value
## (Intercept)        0.9126     0.5080   1.796
## ClimateCold        0.2831     0.5964   0.475
## ClimateTemperate  -0.1684     0.5783  -0.291
## 
## Correlation of Fixed Effects:
##             (Intr) ClmtCl
## ClimateCold -0.852       
## ClimatTmprt -0.878  0.748
\end{verbatim}

The Study\_ID does appear to be showing additional variance but not a
lot still. It also doesn't appear that climate is having an effect on
effect\_Size of nitrate.

\begin{Shaded}
\begin{Highlighting}[]
\NormalTok{mNO3\_TSF }\OtherTok{\textless{}{-}} \FunctionTok{lmer}\NormalTok{(Effect\_size }\SpecialCharTok{\textasciitilde{}}\NormalTok{ TSF }\SpecialCharTok{+}\NormalTok{ (}\DecValTok{1}\SpecialCharTok{|}\NormalTok{Study\_ID), }\AttributeTok{data =}\NormalTok{ effect\_size\_NO3)}

\FunctionTok{plot}\NormalTok{(mNO3\_TSF)}
\end{Highlighting}
\end{Shaded}

\includegraphics{04_Wenas_meta_stats_files/figure-latex/NO3 random effects fire-1.pdf}

\begin{Shaded}
\begin{Highlighting}[]
\FunctionTok{qqnorm}\NormalTok{(}\FunctionTok{resid}\NormalTok{(mNO3\_TSF))}
\end{Highlighting}
\end{Shaded}

\includegraphics{04_Wenas_meta_stats_files/figure-latex/NO3 random effects fire-2.pdf}

\begin{Shaded}
\begin{Highlighting}[]
\FunctionTok{anova}\NormalTok{(mNO3\_TSF) }\CommentTok{\# f{-}value = 0.5874}
\end{Highlighting}
\end{Shaded}

\begin{verbatim}
## Analysis of Variance Table
##     npar Sum Sq Mean Sq F value
## TSF    3 1.4222 0.47406  0.8777
\end{verbatim}

\begin{Shaded}
\begin{Highlighting}[]
\FunctionTok{summary}\NormalTok{(mNO3\_TSF) }
\end{Highlighting}
\end{Shaded}

\begin{verbatim}
## Linear mixed model fit by REML ['lmerMod']
## Formula: Effect_size ~ TSF + (1 | Study_ID)
##    Data: effect_size_NO3
## 
## REML criterion at convergence: 116.4
## 
## Scaled residuals: 
##     Min      1Q  Median      3Q     Max 
## -3.7137 -0.4032  0.0524  0.4738  2.6043 
## 
## Random effects:
##  Groups   Name        Variance Std.Dev.
##  Study_ID (Intercept) 0.3151   0.5613  
##  Residual             0.5401   0.7349  
## Number of obs: 48, groups:  Study_ID, 13
## 
## Fixed effects:
##                Estimate Std. Error t value
## (Intercept)  -4.385e-15  9.248e-01   0.000
## TSF0-1 years  1.040e+00  9.494e-01   1.096
## TSF2-3 years  9.365e-01  9.799e-01   0.956
## TSF4-5 years  4.765e-01  1.031e+00   0.462
## 
## Correlation of Fixed Effects:
##             (Intr) TSF0-y TSF2-y
## TSF0-1years -0.974              
## TSF2-3years -0.944  0.944       
## TSF4-5years -0.897  0.899  0.891
\end{verbatim}

The Study\_ID does appear to be showing additional variance but not a
lot still. Based on the t-values, 0-1 is maybe the most significant but
it is small.

\end{document}
